\section{Example Systems}

	\subsection*{Circle}
	Consider the dynamic system depicted in figure \ref{fig:Circle}
	\begin{equation}\left\{ \begin{aligned}
	\dot{x}_1 &= -a_1 x_1 + b_3 x_3 \\ 
	\dot{x}_2 &= -a_2 x_2 + b_1 x_1 \\ 
	\dot{x}_2 &= -a_3 x_3 + b_2 x_2
	\end{aligned} \right.
	\quad , \quad x_i(0) = x_{i,0} \quad i=1,2,3	
	\label{Circle:eq:model}\end{equation}
	with positive parameters $a_1,\ldots,b_3$ and initial values $x_{i,0}$.
	\begin{figure}[h]
\centering
\includegraphics[scale=0.6]{figures/3VarCircle.png}
\caption{Network Graph of a $3$-states circle.}
\label{fig:Circle}
\end{figure}
	Depending on the magnitude of the parameters, the system could show different 
	qualitative behaviours, it could be
	\begin{enumerate}
	\item stable $|x_i(t) - x_i^*| \to 0 $ as $t\to \infty$ with a fix point $x_i^*$,
	\item periodic $x_i(t)= x_i(t+T)$ with a constant $T$,
	\item unstable $x_i(t) \to \infty$ as $t\to\infty$ or
	\item chaotic. 
	\end{enumerate}
	If we want to to use this system as a simplified version of a real system, e.g. 
	the JAK-STAT pathway \cite{DEN}, we should adjust the parameters in such a way, that 
	the model \eqref{Circle:eq:model} shows a realistic behaviour before we simulate 
	data.\\
	
	\paragraph*{Classical Equilibration}
	The first step in finding a stable behaviour is to compute the fixed points of 
	the system, i.e. $x^*_i$ such that $\dot{x}_i=0$. From \eqref{Circle:eq:model} we
	deduce the homogeneous linear equation system
	\begin{equation}0 = A x \quad \text{where} \quad 
	\begin{pmatrix}
	-a_1 & 0 & b_3 \\ b_1 &-a_2 & 0 \\ 0 & b_2 & -a_3 
	\end{pmatrix} \quad ,
	\end{equation}
	and we know that $\dim \ker A = 0$ if and only if $\det A =0$. 
	So we compute
	\begin{equation}
	\det A = -a_1a_2a_3 + b_1 b_2 b_3
	\end{equation}
	to get: System \eqref{fig:Circle} has only the trivial fixed point $x_i^*=0$, if
	\begin{equation}
	a_1a_2a_3 \neq b_1 b_2 b_3 \quad .
	\end{equation}
	In the case $a_1a_2a_3=b_1b_2b_3$ we get an $1$-dimensional equilibrium manifold
	\begin{equation}
	\ker A = \text{span}\left\{\,\begin{pmatrix}
	b_2 \\ b_1 b_2 \\ a_1 a_2
	\end{pmatrix} \, \right\} \quad .
	\end{equation}
	
	\paragraph*{Tropical Equilibration}
	Now we try to adjust the parameters such that we obtain a approximately stable 
	system using tropical methods \cite{Trop}.\\
	
	Choose a small $1>> \epsilon > 0$ and define normalized parameters (the notation is 
	always understood as $i=1,2,3$)  
	$\tilde{a}_i$ and $\tilde{b}_i$ via
	\begin{equation}\begin{aligned}
	a_1&=\tilde{a}_1 \epsilon^{\gamma_1} \quad , \quad 
	a_2=\tilde{a}_2 \epsilon^{\gamma_2}\quad , \quad 
	a_3=\tilde{a}_3 \epsilon^{\gamma_3}\quad , \quad  \\
	b_1&=\tilde{b}_1 \epsilon^{\gamma_4} \quad , \quad 
	b_2=\tilde{b}_2 \epsilon^{\gamma_5}\quad , \quad 
	b_3=\tilde{b}_3 \epsilon^{\gamma_6}
	\end{aligned}
	\end{equation}	 
	and normalized state variables $\tilde{x}_i$
	\begin{equation}
	x_1=\tilde{x}_1 \epsilon^{\delta_1} \quad , \quad 
	x_2=\tilde{x}_2 \epsilon^{\delta_2}\quad , \quad 
	x_3=\tilde{x}_3 \epsilon^{\delta_3} \quad ,
	\end{equation}
	such that
	\begin{enumerate}
	\item the normalized parameters and variables are of order one\\ 
	$\mathcal{O}(\tilde{a}_i)=\mathcal{O}(\tilde{b}_i)=
	\mathcal{O}(\tilde{x}_i)=\mathcal{O}(1)$ and
	\item the powers are integers $\gamma_i,\delta_i\in \mathbb{Z}$.
	\end{enumerate}
	Note, that in theory we can perform the limit $\epsilon \to 0$ to get
	\begin{equation}
		\lim\limits_{\epsilon \to 0} \tilde{a}_i = \lim\limits_{\epsilon \to 0} 
		\tilde{b}_i = \lim\limits_{\epsilon \to 0} \tilde{x}_i = 1 \quad .
	\end{equation}
	
	We now want to express \eqref{Circle:eq:model} in terms of the normalized parameters 
	and variables. We see that ($i=2,3$ analogous)
	\begin{equation} 
	\left.\begin{aligned}
	\dot{x}_1 &= \dot{\tilde{x}}_1 \epsilon^{\delta_1} \\
	-a_1 x_1 + b_3 x_3 &= -  \tilde{a}_1\tilde{x}_1 \epsilon^{\gamma_1+\delta_1} +
	 \tilde{b}_3 \tilde{x}_3 
	\epsilon^{\gamma_6-\delta_3}	
	 \end{aligned} \right\} \quad \Rightarrow \quad 
	 \dot{\tilde{x}}_1 = 
	 -  \tilde{a}_1\tilde{x}_1 \epsilon^{\gamma_1} +
	 \tilde{b}_3 \tilde{x}_3 
	\epsilon^{\gamma_6+\delta_3-\delta_1}	\label{Circle:eq:normalizedsystem}
	\end{equation}
	and we note that since $0<\epsilon << 1$ the leading summands of 
	\eqref{Circle:eq:normalizedsystem} are those with the smallest power of $\epsilon$.  
	Now tropical equilibration means, we have to find a parameters, such that 
	\begin{enumerate}
	\item the minimal power of $\epsilon$ appears in two summands and 
	\label{Circle:enum_1}
	\item these two leading summands have opposite signs.\label{Circle:enum_2}
	\end{enumerate}
	Now by definition of tropical algebra, \ref{Circle:enum_1}. is just finding the 
	simultaneous tropical roots of
	\begin{equation}\begin{aligned}
		\gamma_1 \oplus \gamma_6 \otimes \delta_3 \otimes (-\delta_1) &= 0 \\
		\gamma_2 \oplus \gamma_4 \otimes \delta_1 \otimes (-\delta_2) &= 0 \\
		\gamma_3 \oplus \gamma_5 \otimes \delta_2 \otimes (-\delta_3) &= 0
	\end{aligned} \label{Circle:eq:tropVar}
	\end{equation}
	or in other words with $f_1(\gamma_1,\ldots,\delta_3) := \gamma_1 \oplus 
	\gamma_6\otimes\delta_3\otimes(-\delta_1)$, $f_2$ and $f_3$ analogous, we have to 	
	compute the tropical variety
	\begin{equation}
		V (f_1,f_2,f_3) \quad .
	\end{equation}
	In this specific case the situation is quiet easy because in 
	\eqref{Circle:eq:normalizedsystem} we only have two summands with opposite signs. 
	Thus the whole tropical variety also fulfils \ref{Circle:enum_2}. Addition 
	of the three equations of \eqref{Circle:eq:tropVar} shows
	\begin{equation}
	V(f_1,f_2,f_3) \subseteq
	\left\{\gamma_1,\ldots,\delta_3 \in\mathbb{Z} \big|
	\gamma_1+\gamma_2 + \gamma_3 = \gamma_4 + \gamma_5 + \gamma_6 \right\} \quad .
	\end{equation}
	To interpret these results in the nontropical world we apply the exponential 
	function and multiply with $\tilde{a}_i$ and $\tilde{b}_i$ (which are all 
	approximately $1$). This yields
	\begin{align}
	\tilde{a}_1\tilde{a}_2\tilde{a}_3 \epsilon^{\gamma_1+\gamma_2+\gamma_3} &=
	\tilde{b}_1\tilde{b}_2\tilde{b}_3 \epsilon^{\gamma_4+\gamma_5+\gamma_6} \notag \\
	\tilde{a}_1\epsilon^{\gamma_1}\tilde{a}_2\epsilon^{\gamma_2}
	\tilde{a}_3\epsilon^{\gamma_3} &=	\tilde{b}_1\epsilon^{\gamma_4} \notag
	\tilde{b}_2\epsilon^{\gamma_5}\tilde{b}_3\epsilon^{\gamma_6}\\
	a_1a_2a_3 &= b_1b_2b_3 \quad .
	\end{align}
	
	We found, that in this example, tropical equilibration leads exactly to 
	those parameter sets, that have a nontrivial classical equilibrium manifold.
	