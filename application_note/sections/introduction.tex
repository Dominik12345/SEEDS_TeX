\section{Introduction}
Cell biology, chemical reactions and mechanical systems are just some examples of 
processes that can be modelled and investigated using dynamic systems. In the mathematical 
formulation these take the form
\begin{equation}
	\dot{x}(t) = f\left(x(t),u(t)\right) \label{eq:f}
\end{equation}
where $x(t)$ is the state of the system at time $t$, $u(t)$ is a known, time-dependent input and 
$f$ is a mathematical model of the system. In addition it is in most cases not possible 
to measure the system state $x$ directly, but only observables $y$ via a observation function 
\begin{equation}
y(t) = h\left(x(t)\right) \quad . \label{eq:h}
\end{equation}
It is more often the rule than the exception, that measurements $y^\text{obs}$ and predictions 
of \eqref{eq:f} and \eqref{eq:h} do not satisfactorily coincide. If the discrepancies are due 
to stochastic processes or measurement noise, the known methods using varieties of Kalman 
Filters and Particle Filters, to name a few, will be able to reconstruct the states $x(t)$ 
given the data $y^\text{obs}$. As soon as the model $f$ contains systematic errors, such as 
unknown interactions, these methods tend to fail or become very unstable. The \textsc{Dynamic 
Elastic Net} (DEN)\cite{DEN} is a dynamic optimization procedure that handles systematic model 
errors as unknown inputs to the systems equations. The methods provided by \textsf{seeds} are 
able to optimize the unknown inputs such to archive a satisfactory fit to given data. 